% !TEX TS-program = XeLaTeX
% !TEX encoding = UTF-8 Unicode

%%%%%%%%%%%%%%%%%%%%%%%%%%%%%%%%%%%%%%%%%%%%%%%%%%%%%%%%%%%%%%%%%%%%%%
%
%	大连理工大学硕士论文 XeLaTeX 模版 —— 格式文件 format.tex
%	版本:0.71
%	最后更新:2010.12.22
%	修改者:Yuri (E-mail: yuri_1985@163.com)
%	编译环境:Ubuntu 10.04 + TeXLive 2010 + TeXworks
%             Windows XP SP3 + CTeXLive 2009 + WinEdt 5.6
%
%%%%%%%%%%%%%%%%%%%%%%%%%%%%%%%%%%%%%%%%%%%%%%%%%%%%%%%%%%%%%%%%%%%%%%

%%%%%%%%%%%%%%%%%%%%%%%%%%%%%%%%%%%%%%%%%%%%%%%%%%%%%%%%%%%%%%%%%%%%%%
% 页面设置
%%%%%%%%%%%%%%%%%%%%%%%%%%%%%%%%%%%%%%%%%%%%%%%%%%%%%%%%%%%%%%%%%%%%%%
% A4 纸张
\setlength{\paperwidth}{21.0cm}
\setlength{\paperheight}{29.7cm}
% 设置正文尺寸大小
\setlength{\textwidth}{16.1cm}
\setlength{\textheight}{22.2cm}
% 设置正文区在正中间
\newlength \mymargin
\setlength{\mymargin}{(\paperwidth-\textwidth)/2}
\setlength{\oddsidemargin}{(\mymargin)-1in}
\setlength{\evensidemargin}{(\mymargin)-1in}
% 设置正文区偏移量,奇数页向右偏,偶数页向左偏
\newlength \myshift
\setlength{\myshift}{0.35cm}	% 双面打印的奇偶页偏移值,可根据需要修改,建议小于 0.5cm
\addtolength{\oddsidemargin}{\myshift}
\addtolength{\evensidemargin}{-\myshift}
% 页眉页脚相关距离设置
\setlength{\topmargin}{-0.05cm}
\setlength{\headheight}{0.50cm}
\setlength{\headsep}{0.90cm}
\setlength{\footskip}{1.47cm}
% 公式的精调
\allowdisplaybreaks[4]  % 可以让公式在排不下的时候分页排,这可避免页面有大段空白。

%下面这组命令使浮动对象的缺省值稍微宽松一点,从而防止幅度
%对象占据过多的文本页面,也可以防止在很大空白的浮动页上放置很小的图形。
\renewcommand{\topfraction}{0.9999999}
\renewcommand{\textfraction}{0.0000001}
\renewcommand{\floatpagefraction}{0.9999}

%%%%%%%%%%%%%%%%%%%%%%%%%%%%%%%%%%%%%%%%%%%%%%%%%%%%%%%%%%%%%%%%%%%%%%
% 字体字号定义
%%%%%%%%%%%%%%%%%%%%%%%%%%%%%%%%%%%%%%%%%%%%%%%%%%%%%%%%%%%%%%%%%%%%%%
% 字号
\newcommand{\yihao}{\fontsize{26pt}{39pt}\selectfont}	  % 一号,1.5  倍行距
\newcommand{\xiaoyi}{\fontsize{24pt}{30pt}\selectfont}  % 小一,1.25 倍行距
\newcommand{\erhao}{\fontsize{22pt}{27.5pt}\selectfont} % 二号,1.25 倍行距
\newcommand{\xiaoer}{\fontsize{18pt}{22.5pt}\selectfont}% 小二,1.25 倍行距
\newcommand{\sanhao}{\fontsize{16pt}{20pt}\selectfont}  % 三号,1.25 倍行距
\newcommand{\xiaosan}{\fontsize{15pt}{19pt}\selectfont} % 小三,1.25 倍行距
\newcommand{\sihao}{\fontsize{14pt}{17.5pt}\selectfont} % 四号,1.25倍行距
\newcommand{\xiaosi}{\fontsize{12pt}{15pt}\selectfont}  % 小四,1.25倍行距
\newcommand{\dawu}{\fontsize{10.5pt}{18pt}\selectfont}  % 五号,1.75倍行距
\newcommand{\zhongwu}{\fontsize{10.5pt}{16pt}\selectfont}% 五号,1.5 倍行距
\newcommand{\wuhao}{\fontsize{10.5pt}{10.5pt}\selectfont}% 五号,单倍行距
\newcommand{\xiaowu}{\fontsize{9pt}{9pt}\selectfont}	   % 小五,单倍行距

\newcommand{\song}{\CJKfamily{song}}
\newcommand{\hei}{\CJKfamily{hei}}
\newcommand{\kai}{\CJKfamily{kai}}
\newcommand{\fs}{\CJKfamily{fs}}
\newcommand{\xkai}{\CJKfamily{xkai}}

% defaultfont 默认字体命令
\def\defaultfont{\renewcommand{\baselinestretch}{1.27}
\fontsize{12pt}{15pt}\selectfont}

% 设置目录字体和行间距
\def\defaultmenufont{\renewcommand{\baselinestretch}{1.22}
\fontsize{12pt}{15pt}\selectfont}

% 固定距离内容填入及下划线
\makeatletter
    \newcommand\fixeddistanceleft[2][1cm]{{\hb@xt@ #1{#2\hss}}}
    \newcommand\fixeddistancecenter[2][1cm]{{\hb@xt@ #1{\hss#2\hss}}}
    \newcommand\fixeddistanceright[2][1cm]{{\hb@xt@ #1{\hss#2}}}
    \newcommand\fixedunderlineleft[2][1cm]{\underline{\hb@xt@ #1{#2\hss}}}
    \newcommand\fixedunderlinecenter[2][1cm]{\underline{\hb@xt@ #1{\hss#2\hss}}}
    \newcommand\fixedunderlineright[2][1cm]{\underline{\hb@xt@ #1{\hss#2}}}
\makeatother

%%%%%%%%%%%%%%%%%%%%%%%%%%%%%%%%%%%%%%%%%%%%%%%%%%%%%%%%%%%%%%%%%%%%%%
% 标题环境相关
%%%%%%%%%%%%%%%%%%%%%%%%%%%%%%%%%%%%%%%%%%%%%%%%%%%%%%%%%%%%%%%%%%%%%%
% 定义、定理等环境
\theoremstyle{plain}
\theoremheaderfont{\hei\bf}
\theorembodyfont{\song\rmfamily}
\newtheorem{definition}{\hei 定义}[chapter]
\newtheorem{example}{\hei 例}[chapter]
\newtheorem{algorithm}{\hei 算法}[chapter]
\newtheorem{theorem}{\hei 定理}[chapter]
\newtheorem{axiom}{\hei 公理}[chapter]
\newtheorem{proposition}[theorem]{\hei 命题}
\newtheorem{property}{\hei 性质}
\newtheorem{lemma}[theorem]{\hei 引理}
\newtheorem{corollary}{\hei 推论}[chapter]
\newtheorem{remark}{\hei 注解}[chapter]
\newenvironment{proof}{\hei{证明} }{\hfill $\square$ \vskip 4mm}

% 目录标题
\renewcommand\contentsname{\hfill\LARGE 目  录 \hfill}
\renewcommand\listfigurename{\hfill 插~图~目~录 \hfill}
\renewcommand\listtablename{\hfill 表~格~目~录 \hfill}
\renewcommand{\bibname}{\hfill 参~考~文~献 \hfill}

%%%%%%%%%%%%%%%%%%%%%%%%%%%%%%%%%%%%%%%%%%%%%%%%%%%%%%%%%%%%%%%%%%%%%%
% 段落章节相关
%%%%%%%%%%%%%%%%%%%%%%%%%%%%%%%%%%%%%%%%%%%%%%%%%%%%%%%%%%%%%%%%%%%%%%
\setcounter{secnumdepth}{4}
\setcounter{tocdepth}{4}
% 设置章、节、小节、小小节的间距
\titleformat{\chapter}[hang]{\normalfont\xiaosan\hei\sf}{\xiaosan\thechapter}{10pt}{\xiaosan}
\titlespacing{\chapter}{0pt}{-3ex  plus .1ex minus .2ex}{3.3ex}
\titleformat{\section}[hang]{\sihao\hei\sf}{\sihao\thesection}{0.5em}{}{}
\titlespacing{\section}{0pt}{0.5em}{0.5em}
\titleformat{\subsection}[hang]{\xiaosi\hei\sf}{\xiaosi\thesubsection}{0.5em}{}{}
\titlespacing{\subsection}{0pt}{0.5em}{0.3em}
\titleformat{\subsubsection}[hang]{\hei\sf}{\thesubsubsection}{0.5em}{}{}
\titlespacing{\subsubsection}{0pt}{0.3em}{0pt}
% 缩小目录中各级标题之间的缩进
\dottedcontents{chapter}[0.32cm]{\vspace{0.2em}}{1.0em}{5pt}
\dottedcontents{section}[1.32cm]{}{1.8em}{5pt}
\dottedcontents{subsection}[2.32cm]{}{2.7em}{5pt}
\dottedcontents{subsubsection}[3.32cm]{}{3.4em}{5pt}

% 段落之间的竖直距离
\setlength{\parskip}{1.2pt}
% 段落缩进
\setlength{\parindent}{24pt}
% 定义行距
\renewcommand{\baselinestretch}{1.27}

%%%%%%%%%%%%%%%%%%%%%%%%%%%%%%%%%%%%%%%%%%%%%%%%%%%%%%%%%%%%%%%%%%%%%%
% 页眉页脚设置
%%%%%%%%%%%%%%%%%%%%%%%%%%%%%%%%%%%%%%%%%%%%%%%%%%%%%%%%%%%%%%%%%%%%%%

\newcommand{\makeheadrule}{%
    \makebox[0pt][l]{\rule[.7\baselineskip]{\headwidth}{0.5pt}}%
    \vskip-.8\baselineskip}

\makeatletter
\renewcommand{\headrule}{%
    {\if@fancyplain\let\headrulewidth\plainheadrulewidth\fi
     \makeheadrule}}

\pagestyle{fancyplain}

\fancyhf{}
\fancyhead[CO]{\song\wuhao{辽宁工程技术大学硕士学位论文}}
\fancyhead[CE]{\song\wuhao\@ctitle}
\fancyfoot[C,C]{\xiaowu$-$~\thepage~$-$}

% Clear Header Style on the Last Empty Odd pages
\makeatletter
\def\cleardoublepage{\clearpage\if@twoside \ifodd\c@page\else%
    \hbox{}%
    \thispagestyle{empty}%              % Empty header styles
    \newpage%
    \if@twocolumn\hbox{}\newpage\fi\fi\fi}



%%%%%%%%%%%%%%%%%%%%%%%%%%%%%%%%%%%%%%%%%%%%%%%%%%%%%%%%%%%%%%%%%%%%%%
% 列表环境设置
%%%%%%%%%%%%%%%%%%%%%%%%%%%%%%%%%%%%%%%%%%%%%%%%%%%%%%%%%%%%%%%%%%%%%%
\let\orig@Itemize =\itemize
\let\orig@Enumerate =\enumerate
\let\orig@Description =\description

\def\Myspacing{\itemsep=1ex \topsep=-4ex \partopsep=-2ex \parskip=-1ex \parsep=2ex}
\def\newitemsep{
\renewenvironment{itemize}{\orig@Itemize\Myspacing}{\endlist}
\renewenvironment{enumerate}{\orig@Enumerate\Myspacing}{\endlist}
\renewenvironment{description}{\orig@Description\Myspacing}{\endlist}
}
\def\olditemsep{
\renewenvironment{itemize}{\orig@Itemize}{\endlist}
\renewenvironment{enumerate}{\orig@Enumerate}{\endlist}
\renewenvironment{description}{\orig@Description}{\endlist}
}
\renewcommand{\labelenumi}{(\arabic{enumi})}
\newitemsep

%%%%%%%%%%%%%%%%%%%%%%%%%%%%%%%%%%%%%%%%%%%%%%%%%%%%%%%%%%%%%%%%%%%%%%
% 其他设置
%%%%%%%%%%%%%%%%%%%%%%%%%%%%%%%%%%%%%%%%%%%%%%%%%%%%%%%%%%%%%%%%%%%%%%
% 增加 \ucite 命令使显示的引用为上标形式
\newcommand{\ucite}[1]{$^{\mbox{\scriptsize \cite{#1}}}$}

%%%%%%%%%%%%%%%%%%%%%%%%%%%%%%%%%%%%%%%%%%%%%%%%%%%%%%%%%%%%%%%%%%%%%%
% 图形表格
%%%%%%%%%%%%%%%%%%%%%%%%%%%%%%%%%%%%%%%%%%%%%%%%%%%%%%%%%%%%%%%%%%%%%%
\renewcommand{\figurename}{图}
\renewcommand{\tablename}{表}
%\captionstyle{\centering}
%\hangcaption
\captiondelim{\hspace{1em}}
\captionnamefont{\song\rmfamily\zhongwu\selectfont}
\captiontitlefont{\song\rmfamily\zhongwu\selectfont}

\newcommand{\tablepage}[2]{\begin{minipage}{#1}\vspace{0.5ex} #2 \vspace{0.5ex}\end{minipage}}
\newcommand{\returnpage}[2]{\begin{minipage}{#1}\vspace{0.5ex} #2 \vspace{-1.5ex}\end{minipage}}


%%%%%%%%%%%%%%%%%%%%%%%%%%%%%%%%%%%%%%%%%%%%%%%%%%%%%%%%%%%%%%%%%%%%%%
% 定义题头格言的格式
%%%%%%%%%%%%%%%%%%%%%%%%%%%%%%%%%%%%%%%%%%%%%%%%%%%%%%%%%%%%%%%%%%%%%%

\newsavebox{\AphorismAuthor}
\newenvironment{Aphorism}[1]
{\vspace{0.5cm}\begin{sloppypar} \slshape
\sbox{\AphorismAuthor}{#1}
\begin{quote}\small\itshape }
{\\ \hspace*{\fill}------\hspace{0.2cm} \usebox{\AphorismAuthor}
\end{quote}
\end{sloppypar}\vspace{0.5cm}}

%自定义一个空命令,用于注释掉文本中不需要的部分。
\newcommand{\comment}[1]{}

% This is the flag for longer version
\newcommand{\longer}[2]{#1}

\newcommand{\ds}{\displaystyle}

% define graph scale
\def\gs{1.0}

\def\acknowledgements{
     \newpage
     \thispagestyle{empty}
     \chapter*{\hfill \LARGE 致  谢 \hfill}
\addcontentsline{toc}{chapter}{致  谢}
时光飞逝,短暂的硕士研究生生涯临近结束。值此论文即将完成之际,
向所有关心、指导、帮助过我的老师、朋友、同学们,表示衷心的感谢。

首先向两年多来在我的学业和做人想给予了指导和关怀的导师赵国忱教授
表示衷心的感谢。两年多来导师以他渊博的学识对我的学习、研究给予了悉心的指导,
提供了极佳的环境,营造了自由的氛围。本文从选题、撰写、修改到最终定稿,
都倾注了导师的大量心血。他严谨求实的治学态度、宽阔的视野、
创新而敏锐的思维、勇于进取、甘当人梯的精神,
严于律己、宽以待人的思想品质使我深感敬佩,并让我受益终生。

感谢辽宁工程技术大学测绘与地理科学学院两年多来的培养,感谢学院的
领导与老师们的关系、帮助和支持。

感谢鄂尔多斯昊华精煤有限公司技术科的全体同志,他们精湛的技术、
高标准的工作、随和的待人态度,使在和他们一起工作的这段时间给我留下了
美好、深刻的印象。

感谢和我在同一个宿舍生活过的同学:邹阳、徐杨、吴作启、赵亮
他们容忍了几乎黑白颠倒的生活,忍受了我向总是向他们说些他们并不关心的事情。
更重要的是,和他们一起无论是讨论学术还是生活问题,总能让我受益匪浅。
感谢范良同学,我们在一起的合作,同样让我学到了非常多。

感谢两年多的研究生生活中,班级同学对我的关心和帮助。

感谢参考文献中提到的各文献的作者所提供的研究思路与方法,没有他们开创性的研究,
就不会有这篇论文,在此对他们表示最崇高的敬意与感谢。

}


\raggedbottom 
