% !TEX TS-program = XeLaTeX
% !TEX encoding = UTF-8 Unicode

\chapter{工程实例}

\label{chap04}

\section{工程背景} 
内蒙古自治区鄂尔多斯市某煤矿是一座新建的年产千万吨的现代化煤矿。
从2010年开始试运营。配有完善的生产生活设施,有完善的生产生活条件,
有洗煤厂,并有一条铁路再建。
为了在生产、加工、运输、销售等环节间进行有效的缓存,共建有原煤仓
两个,矸石仓两个、块煤仓4个、末煤仓2个,其中,以两个末煤仓最大
装载能力可以达到2.5万吨。
               
                  插入矿区效果图


\section{观测方法}
由上一节可以看到,装载量上万吨的煤仓将会对地面产生巨大的压力。
为了保证煤仓石使用的安全,研究煤仓沉降的规律。需要对其进行沉降观测。
本工程使用Leica DNA03 电子水准仪进行。
首先在观察区域附近布设数个基准点,并定期和矿区外控制点按照
国家二等水准测量连测。
每个煤仓上有四个控制点,绕圈观测。。。三等。

\section{数据预处理}
数据以Laica *** 格式导出,使用Leica Office获得各个观测点的高程值。
生成一个逗号分割的文本文件(csv)。
通过这个csv文件,生成html沉降值报表和PNG格式的图像(gnuplot)。
生成HTML格式的沉降、倾斜状况报表。


\section{FastICA}
\subsection{直接使用FastICA}
\subsection{首先使用PCA降维再应用FastICA}
\subsection{两种方法的对比}
