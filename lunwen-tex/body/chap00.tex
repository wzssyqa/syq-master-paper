% !TEX TS-program = XeLaTeX
% !TEX encoding = UTF-8 Unicode

\chapter*{\hfill 引  言 \hfill}
\addcontentsline{toc}{chapter}{引  言}
\label{chap00}

\section{选题的背景与意义}
随着社会的发展,人类的生产和生活对能源的需求越来越大。在积极开发新能源的同时,
必须看到,当前,尤其是我国对传统化石能源的依赖仍然非常大,特别是煤炭。
更加规范的管理煤炭的生产、运输、使用等在对于安全生产、保护环境等是非常重要的。
其中,煤仓是煤炭的生产、运输、使用等环节中都非常重要的设施,其可以作为各环节
之间的缓冲,也可以避免煤炭露天堆放产生的消耗和环境污染,因此在煤矿、煤炭销售、
热电厂、煤化工厂等中都得到了广泛的使用。

作为一种大型的工业建筑,其直径甚至可以达到百余米(***),高可达数十米,装载能力可以
达到数十万吨。如此巨大的建筑、如此多的载荷将是对设计、施工和运营的巨大考验,
也是对地基和煤仓壁的巨大考验。

对于此类大型建筑进行变形监测是非常重要,也是非常必要的。历史上,曾经有很多的教训(***):
\begin{asparaitem}[$\bullet$]
\item 法国67米高的马尔巴塞(Malpasset)拱坝1959年垮坝
\item 意大利262米高的瓦依昂(Vajoint)拱坝1963年因库岸大滑坡导致涌浪翻坝且水库淤满失效
\item 美国93米高的提堂土坝1976年溃决
\item 我国板桥和石漫滩两座土坝1975年洪水漫坝失事
\end{asparaitem}
当然也有非常多因为得力的监测工作避免或者降低生命财产损失的案例(***):
\begin{asparaitem}[$\bullet$]
\item 1985年6月12日长江三峡新滩大滑坡的成功预报成功确保灾害损失减少到了最低限度
\item 隔河岩大坝外观变形GPS自动化监测系统在1998年长江流域抗洪错峰中所发挥的巨大作用,确保和安全渡汛,避免了荆江大堤灾难性的分洪
\end{asparaitem}

安全监测除了及时掌握建筑物的工作性态,确保其安全外,还有多方面的必要性。
美国肯务局认为,(***)
对建筑物及地基进行长期和系统的监测,是诊断、预测、法律和研究等四个方面的需要:
\begin{asparaitem}[$\bullet$]
\item 诊断的需要。包括验证设计参数与改进设计,对新施工技术的优越性进行评价;
对不安全迹象和险情进行诊断并采取措施予以加固;以及验证建筑物运行是否处于持续正常状态。
\item 预测的需要。利用长期积累的观测资料掌握变化规律,对建筑物的未来性态做出及时有效的预报。
\item 法律的需要。对由于工程事故而引起的责任和赔偿问题,观测资料有助于确定其原因和责任,以便法庭做出公正判决。
\item 研究的需要。观测资料是建筑物工作性态的真实反映,为未来设计提供定量信息,
可改进施工技术,利于设计概念的更新和对破坏机理的了解。
\end{asparaitem}

煤仓、大坝等工业建筑与其它民用建筑相比,有其特殊的地方,包括:
\begin{asparaitem}[$\bullet$]
\item 建设阶段相比运营阶段所受的压力要小很多,这样,其它一些建筑在建设时就会表现出来的问题,
对于这些建筑可能要到运营一段时间之后才能表现出来。
\item 运营阶段受压力变化会很大,比如,水库可能干涸,煤仓可能出现空仓,
这种变化是否会对建筑的稳定产生巨大影响需要特别关注,也会对数据的处理提出更高的要求。
\end{asparaitem}

\section{现有的观测手段和数据处理方法}
如上所述,在施工和运营阶段对于建筑物进行变形监测是非常重要和必要的。
为了解决这个更好地对建筑物的形变进行观测和预报,掌握建筑物形变的规律,
人们采取了非常多的措施来观测,引入了非常多的数据处理方法。

\subsection{建筑物变形监测的方法}
随着技术的发展,特别是通信技术、传感器技术等的发展,很多新的测量手段引入的
变形监测领域。随着技术的继续发展,我们相信还会有更多的新测量手段引入。
\subsubsection{常规的大地测量方法}
常规的大地测量方法是指利用常规大地测量仪器,测量方向、角度、边长、高差等量
所采用的方法的总称,包括布设成网形来确定一维、二维、三维坐标的网平差法、
各种交会法、极坐标法卫星定位法,以及几何水准、三角高程法等。(***)
常规的大地测量仪器有光学经纬仪,光学水准仪、电磁波测距仪、电子经纬仪、
电子水准仪、电子全站仪以及GNSS接收机等。测量方法的选择,涉及到测量误差的
来源、产生、传播以及改正消除方法等方面的知识,需要精心设计和论证。(***)
\subsubsection{摄影测量方法}
与其它方法相比,摄影策略方法有下述显著特点,可在某些变形监测中应用(***):
\begin{asparaitem}[$\bullet$]
\item 不需要接触被监测的变形体。
\item 外业工作量小,观测时间短,可获取快速变形过程,可同时确定变形体上任意点的变形。
\item 摄影影像的信息量大,利用率高,利用种类多,可以对变形前后的信息做各种后处理,通过底片可观测到变形体任一时刻的状态。
\item 缺点是,摄影仪器的费用较高,数据处理对软硬件的要求也较高。
\end{asparaitem}
摄影测量方法的精度(***)主要取决于像点坐标的测量精度和摄影测量的几何强度。
前者与摄影机和量测仪的质量、摄影材料质量有关,后者与摄影站和变形体之间的关系
以及变形体上控制点的数量和分布有关。在数据处理中采用严密的光束法平差,
将外方位元素、控制点的坐标以及摄影测量中的系统误差如底片形变、镜头畸变作为
观测值或估计参数一起进行平差,亦可以进一步提高变形体上被测目标点的精度。
目前摄影测量的硬件和软件发展很快,相片坐标精度可达2~4${\mu}$m,目标点精度
可达摄影距离的${1/10^5}$。近年来发展起来的数字摄影测量和实时摄影测量技术
在变形监测中有更好的应用前景。
\subsubsection{测量短距离及其变化的方法}
由于电磁波测距法受固定误差的限制,对于小于$50m$的的距离策略不宜采用这种方法。
根据实际条件可以采用机械法,如Gerick研制的金属丝测长仪,是将很细的金属丝在
固定拉力下绕在因瓦测鼓上,其优点是受温度影响小,在上述测程下可达优于$1mm$的精度。(***)
\subsubsection{准直法}
水平基准线通常平行于被监测物体,如大坝、机器设备的轴线。偏离基准线距离或到基准线所构成的
垂直基准面的偏离值称为偏距(或垂距),测量偏距的过程称为准直测量。基准线(或基准面)
可用光学法、光电法和机械法产生。(***)
\subsubsection{铅直法}
以过基准点的铅垂线为垂直基准线,沿铅垂基准线的目标点相对于铅垂线的水平距离(亦称偏距)
可通过垂线坐标仪、测尺或传感器得到。与准直法一样,铅垂线可以用光学法、光电法或机械法产生。
机械法主要是可以克服风和摆动的影响,最常用的机械法是正、倒垂线法。(***)
\subsubsection{液体静力水准法}
基于贝努利方程,即对于连通管中处于静止状态的液体压力,满足$P+{\rho}gh=C$,
其中$P$为空气压力,$\rho$为液体密度,$g$为重力加速度,$h$为液体水柱高,$C$为一常数。
按此原理制成的液体静力水准测量仪或系统可以观测两点或多点之间的高差。若其中的一个观测头
安装在基准点上,其它观测头安装在目标点上,进行多期观测,则可得个目标点的垂直位移。
这种方法特别适合建筑物内部(如大坝)的沉降观测,尤其是适用那些使用常规的光学水准法观测
较困难且高差又不太大的情况。目前液体经历水准测量系统采用自动读数装置,可实现持续监测,
监测点可达上百个。此外还发展了移动式系统,观测的高差可达数米,
因此也可用于桥梁的沉降变形监测。(***)
\subsubsection{挠度测量}
挠度曲线为相对于水平线或铅垂线(称基准线)的弯曲线,在曲线上某点到基准线的距离称为挠度。
大坝的挠度曲线及其随时间的变化可以通过正、倒垂线法或倾斜测量方法获得。
房屋类高层建筑物的挠度可以由观测不同高度处的倾斜来换算求得。(***)
\subsubsection{倾斜观测}
两点之间的倾斜可以使用测斜仪直接测得。测斜仪包括摆式测斜仪、
伺服加速度计式测斜仪以及电子水准器等。采用电子测斜仪可进行动态观测。
两点之间的倾斜也可以用测量高差或水平位移,通过两点间距离间接获得。(***)
\subsubsection{裂缝的观测方法}
工程建筑物的裂缝观测内容包括对裂缝编号,观测裂缝的位置、走向、长度、宽度等。
对于重要的裂缝,要埋设观测标志,用游标卡尺定期地测定两个标志头之间距离的变化,
确定裂缝的发展情况。
对于建筑预留缝和岩石缝隙这种更小距离的测量,一般通过预埋内部测微计和外部测微计进行。
测微计通常由金属丝或因瓦丝与测表构成,其精度可优于$1mm$。(***)
\subsubsection{振动的观测方法}
对于塔式建筑物,在温度和锋利荷载作用下会来回摆动,从而就需要对建筑物进行动态观测,
即震动(摆动)观测。有的桥梁也需要进行震动观测,对于特高的房屋建筑,也存在振动现象。
观测建筑物的振动,可采用专门的光电观测系统,其原理与激光铅直相似,也可以采用GNSS
技术做持续的动态振动观测。(***)
\subsubsection{三维激光扫描测量法}
三维激光扫描仪在不同位置对被测对象进行扫描得到“点云”,通过后处理软件可获取物体
在给定的坐标系下的三维坐标。车载、机载激光扫描仪将成为21世纪地面火速据采集的
主要手段,已经被广泛应用在工程建筑和变形监测方面,并将成为一种广泛应用的、
重要的变形监测方法。(***)
\subsubsection{变形监测的自动化}
变形监测的自动化要求基于下述原因:
\begin{asparaitem}[$\bullet$]
\item 变形速度太快
\item 监测点太多——需要同一时刻获得许多个测点上的变形;
\item 监测间隔太短——变形过程需要大量短时间间隔的观测数据描述;
\item 监测环境太恶劣——噪声、高压、高热、高磁场或人无法到达;
\item 监测不能影响生产和运行管理。
\end{asparaitem}
基于信号转换的传感技术,可以把变形监测中需要确定的距离、角度、高差、倾角等几何量
及其微小变化转化为点信号。将这些用于变形监测以及精密测量的传感器安装在伸缩仪、
应变仪、准直仪、铅直仪、测斜仪以及经历水准测量系统中,通过数据获取、信号处理、
数据转换与通信,可将成百上千个测点上的监测数据传送到数据终端或数据处理中心,
实现变形的持续监测、数据的自动记录、传输与处理。(***)

\subsection{变形分析方法(***)}
变形分析的研究内容涉及到变形数据的分析、变形物理解释和变形预报的各个方面,
通常可将其分为变形的几何分析和变形的物理解释两部分。变形的几何分析是对变形体的
形状和大小的变形做几何描述,其任务在于描述变形体变形的空间状态和时间特性。
变形物理解释的任务是确定变形体的变形和变形原因之间的关系,解释变形的原因。

\subsubsection{传统变形的时空特征分析机器建模方法}
传统的变形几何分析主要包括参考点的稳定性分析、观测值的平差处理和质量评定以及
变形模型参数估计等内容。
\paragraph*{参考点的稳定性分析} 
监测点的变形信息是想对于参考点或一定基准的,如果所选基准本省不稳定或不统一,
则由此获得的变形值就不能反映真正意义上的变形,因此,变形的基准问题是变形监测
数据处理首先必须考虑的问题。过去对参考点的稳定性分析研究主要局限于周期性的监测网,
其方法有很多,例如,A. Chrzanowski(***)论述的这样的5种方法:
\begin{asparaitem}[$\bullet$]
\item 以方差分析进行整体检验为基础的的Hannover法(***),即“平均间隙法”
\item 以$B$检验法为基础的Delft法,即单点位移分量法(***)
\item 以方差分析和点的位移想了为基础的Karlsruhe法(***)
\item 考虑大地基准的Munich法(***)
\item 以位移的不变函数分析为基础的Fredericton法(***)
\end{asparaitem}
后来又发展了稳健-$S$法,也称逐次定权迭代法。(***)
\paragraph*{观测值的平差处理和质量评定}
观测值的质量好坏直接关系到变形值的精度和可靠性。在这方面主要涉及到观测值质量、
平差基准、粗差处理、变形的可区分性等几项内容。在固定基准的经典平差基础上,
发展了重心基准的自由网平差和拟稳基准的拟稳平差(***)。在W. Baarda提出数据探测法(***)
后,粗差探测与变形的可区分性研究成果已经极为丰富,这以体现在李德仁(***)、黄幼才(***)、
陶本藻(***)等的著作中。
\paragraph*{变形模型参数估计}
陈永奇(***)概括了两种基本的分析方法,即直接法和位移法。直接法是利用原始的重复观测值
之差计算应变分量或它们的变化率;位移法是利用各观测点坐标的平差值之差(位移值)计算应变分量。
同时他还提出了变形分析通用法,研制了相应的软件DEFNAN。

\subsubsection{动态变形分析}
实质上,自从20世纪70年代末至90年代初,对几何变形分析研究得较为完善的是用常规地面测量
技术进行周期性监测的静态模型,但它考虑的仅仅是变形体在不同观测时刻的空间状态,并没有
很好的建立各个状态之间的联系,更谈不上变形监测自动化系统的变形分析研究。事实上,
变形体在不同状态之间是具有时间关联性的。为此,后来许多学者转向了对时序观测数据的
动态模型研究,如变形的时间序列分析方法建模;基于数字信号处理的数字滤波技术分离时效分量;
变形的卡尔曼滤波模型;用FIR(Finite Impulse Response)滤波器抑制GPS多路径效应等。(***)

