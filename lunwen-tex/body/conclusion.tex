% !TEX TS-program = XeLaTeX
% !TEX encoding = UTF-8 Unicode

\chapter*{\hfill 结  论 \hfill}
\addcontentsline{toc}{chapter}{结  论}

本文首先分析了现有建筑物变形的观测方法与数据处理方法;分析了煤仓相对于其它建筑物的特点;
介绍了盲信号处理及独立成分分析,指出了其使用情景以及限制因素;介绍了相关系数、互信息、
灰关联度三种相关关系的表征值。

本文布设了沉降观测网控制网,使用二等水准测量对煤仓进行了沉降观测,
编写程序进行了数据分析,给出了沉降观测报表并绘制了沉降观测曲线。
整个监测过程中,煤仓总的沉降量最大为36毫米,最大相对倾斜也远小于规定的1/2000。
这些都表明煤仓的运营是安全的。

通过使用独立成分分析算法分析观测的煤仓沉降数据,对比各环境因素对煤仓沉降,
分析它们对沉降的影响,可以看到其规律基本符合我们对结果的预期,即
煤仓的装载量与某个分量呈现明显的相关关系,而与其它分量无明显的正相关或负相关关系,
当地温度和使用时间与各个分量也无明显正相关或负相关关系。

但是由于观测的时间还比较短,可能与实际情况有较大差异,需要更多数据来确定这种关系,
以上分析只能是定性分析,还不能给出各个因素对沉降的具体影响的大小;
类似的工程测量项目可以获取的数据组数要明显少于语音、图像识别等应用,我们更加需要一种可靠的对数据加密的方法;
使用的线性的独立成分分析,得出的结论可能与实际情况不完全相符合;
一个工程的并不能证明这种方法可以适用于同类工程,我们需要更多类似工程的数据来验证这种方法的适用性。
因此需要进一步对该煤仓的沉降进行跟踪观测;对数据进行定量分析;
使用非线性独立分量分析代替线性的独立分量分析,进而获得沉降量与各种影响因素的定量关系。
